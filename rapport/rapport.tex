%Classe du Document% 
\documentclass[a4paper,french,12pt]{report}


%Packages utilisés
\usepackage[latin1]{inputenc}
\usepackage[T1]{fontenc}
\usepackage[francais]{babel}
%\usepackage{layout}
%\usepackage{geometry}
\usepackage{setspace}
\usepackage{fixltx2e}
%\usepackage{soul}
%\usepackage{ulem}
%\usepackage{eurosym}
\usepackage{graphicx}
%\usepackage{bookman}
%\usepackage{charter}
%\usepackage{newcent}
%\usepackage{lmodern}
%\usepackage{mathpazo}
%\usepackage{mathptmx}
%\usepackage{url}
%\usepackage{verbatim}
%\usepackage{moreverb}
%\usepackage{listings}
%\usepackage{fancyhdr}
\usepackage{wrapfig}
%\usepackage{color}
%\usepackage{colortbl}
%\usepackage{amsmath}
%\usepackage{amssymb}
%\usepackage{mathrsfs}
%\usepackage{asmthm}
%\usepackage{makeidx}

\usepackage{hyperref}
\hypersetup{
	bookmarks=true,
	colorlinks=true,
	linkcolor=black, 
}

\begin{document}

\definecolor{mail}{rgb}{0.2156862745,0.862745098,0.3215686275} %Couleur des adresses mails

\title{R�alisation d'un panorama}

\author{
	\bsc{Ludovic Lamarche} \\ 
	\bsc{Quentin Perales} \\ 
	\bsc{Elie Poussou} \\ \\ \\ \\
	E.I.S.T.I\\ 
	PAU} 
\date{13 Octobre 2013}


\maketitle
\tableofcontents

%\singlespacing
\onehalfspacing
%\doublespacing 


	\chapter*{Introduction}
	\chapter{Avancement du projet}
		\section{Premiers pas dans le projet}
		\section{Ce qu'il reste à faire}
	\chapter{Les différentes façons de faire un panorama}
		\section{Pr�-assemblage}
		\section{Premi�re méthode : Point par point}
		\section{Seconde méthode : Les lignes}
	%	\section{Troisième méthode : Correspondance des distances}
		\section{Post-assemblage}
	\chapter*{Conclusion}
\end{document}
