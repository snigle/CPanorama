%Classe du Document% 
\documentclass[a4paper,french,12pt]{report}


%Packages utilisés
\usepackage[latin1]{inputenc}
\usepackage[T1]{fontenc}
\usepackage[francais]{babel}
%\usepackage{layout}
%\usepackage{geometry}
\usepackage{setspace}
\usepackage{fixltx2e}
%\usepackage{soul}
%\usepackage{ulem}
%\usepackage{eurosym}
\usepackage{graphicx}
%\usepackage{bookman}
%\usepackage{charter}
%\usepackage{newcent}
%\usepackage{lmodern}
%\usepackage{mathpazo}
%\usepackage{mathptmx}
%\usepackage{url}
%\usepackage{verbatim}
%\usepackage{moreverb}
%\usepackage{listings}
%\usepackage{fancyhdr}
\usepackage{wrapfig}
%\usepackage{color}
%\usepackage{colortbl}
%\usepackage{amsmath}
%\usepackage{amssymb}
%\usepackage{mathrsfs}
%\usepackage{asmthm}
%\usepackage{makeidx}

\usepackage{hyperref}
\hypersetup{
	bookmarks=true,
	colorlinks=true,
	linkcolor=black, 
}

\begin{document}



\title{R�alisation d'un panorama}

\author{
	\bsc{Ludovic Lamarche} \\ 
	\bsc{Quentin Perales} \\ 
	\bsc{Elie Poussou} \\ \\ \\ \\
	E.I.S.T.I\\ 
	PAU} 
\date{13 Octobre 2013}


\maketitle
\tableofcontents

%\singlespacing
\onehalfspacing
%\doublespacing 

	\chapter*{Introduction}	
	    Le projet de premier semestre de classe ING1 a pour but d'automatiser la r�alisation d'un panorama, c'est � dire d'assembler des images qui s'entre-coupent pour r�aliser une image g�n�rale. G�n�ralement, ce mode de photo est utilis� par les photographes pour obtenir la photo � grande �chelle d'un paysage.\\
	    Nous sommes r�partis par groupe de 3, Ludovic, Elie et Quentin.\\
	    Pour le premier livrable, nous devons faire un programme en C qui sauvegarde et charge une image pour apprendre les bases du langage, et se familiariser avec le C.\\
	    Ensuite, il faut faire des recherches pour comprendre comment r�aliser le panorama, et d�gager des m�thodes pour parvenir � automatiser sa cr�ation.
	\chapter{Avancement du projet}
		\section{Premiers pas dans le projet}
		    Dans un premier temps, nous avons cherch� comment coder en C, fare des recherches plus pous�es sur le langage apr�s les cours, comprendre comment les images �taient con�ues et apprendre � manipuler les fichiers en g�n�ral.\\
		    Apr�s quelques essais nous avons con�u la charge et la sauvegarde de l'image avec un tableau d'une seule dimension. Puis, pour faciliter l'utilisation future de l'image, nous avons r�aliser une charge et sauvegarde pour une matrice, ce qui permet de visualiser facilement la photo de base.\\
		    Pour tester ces fonctions, nous avons �crit une fonction qui permet de passer d'une image couleur � une image en teinte de gris comme demand�e pour le livrable 2.\\
		    Les teintes noires et banches sont donn�es par l'�quation suivante : teinteNoire = teinteR + teinteG + teinteB.\\
		\section{Ce qu'il reste � faire}
		    Cependant, tout le code du second livrable ainsi que celui du panorama entier reste � faire m�me si la r�flexion sur le code a �t� importante au cours de ce premier mois de travail.\\
	\chapter{Les diff�rentes fa�ons de faire un panorama}
		\section{Pr�-assemblage}
		\section{Premi�re m�thode : Point par point}
		\section{Seconde m�thode : Les lignes}
	%	\section{Troisi�me m�thode : Correspondance des distances}
		\section{Post-assemblage}
	\chapter*{Conclusion}
\end{document}
